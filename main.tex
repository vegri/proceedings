\documentclass{PoS}

\let\ifpdf\relax

\usepackage{cite}
\usepackage[square, comma, numbers, sort&compress]{natbib}
\usepackage{wrapfig}
\usepackage[utf8]{inputenc}
\usepackage[T1]{fontenc}
% \usepackage{lmodern}
\usepackage{amsfonts}
\usepackage{amsmath}
\usepackage{graphicx}
\usepackage{booktabs}
\usepackage{placeins}
% \usepackage{pdfpages}
\usepackage{caption}
\usepackage{multirow}
\usepackage{hhline}
\usepackage{capt-of}
\usepackage{array}
\usepackage{subcaption}
\usepackage{here}
\usepackage[labelfont=bf, format=plain, font=it]{caption}
\usepackage[german]{babel}
% \usepackage{biblatex}

\parindent0pt
\hyphenation{Ex-pe-ri-ment}

\title{Spurdetektoren} \author{\speaker{Verena
Grimm}\\
        Johannes Gutenberg Universität, 55099 Mainz\\
        \\
        E-mail: \email{vegrimm@students.uni-mainz.de}}

\abstract{
% The history of cold fusion is regarded under the light of the scientific method.
% A short introduction to nuclear fusion in general is given before having a look at the beginnings of cold
% fusion in the 1940's, leading to the most famous and controversially discussed experiment of cold fusion
%  of Fleischmann and
% Pons in 1989. Finally, the research in the years following as well as its future prospects are briefly
% discussed, concluding the necessity of further research.
einleitung in spurdetektoren (möglichkeiten, spuren geladener teilchen zu detektieren; warum macht
man das; grundlegende funktionen von gas, halbleiter und szintidetektoren werden kurz erläutert und
vor-und nachteile sowie einsatzgebiete 

}

\FullConference{Masterseminar I: Thema -- Kern- und Teilchenphysik\\
		 WS 2014/2015,\\
		 Mainz, Germany}
\ShortTitle{Spurdetektoren}
% \PoS{Keine Ahnung was das ist}

\begin{document}

\section{Einleitung: Wozu die Spur eines Teilchens bestimmen?}

Teilchendetektoren dienen dem Nachweis freier Teilchens durch Messung verschiedener Parameter wie
z.B. Energie, Impuls oder Ladung. Dabei wird die absorbierte Energie, die ein Teilchen beim
Durchgang durch die Materie des Detektors an diese abgibt, in ein auswertbares Signal umgewandelt.
Ziel eines Spurdetektors ist es, mit diesem Signal auch eine Ortsinformation zu erhalten. Somit
kann man die Trajektorie eines Teilchen durch Detektion an verschiedenen Orten und anschließender
Rekonstruktion der Bahnkurve bestimmen.
\\
Mit der Spur des Teilchens kann man dessen Ursprung zurückverfolgen, bei einem Collider-Experiment
beispielsweise wäre dies der Ort der Kollision der Primärteilchen. Dies kann man zur
Untergrundunterdrückung - falls das Teilchen von irgendwo anders herstammt - und auch zur
Messung von Zerfallsstrecken einsetzen, bei der dann die Bahnkurve der Teilchen bestimmt wird, in
die ein Sekundärteilchen zerfallen ist. Viel wichtiger ist noch, bei vielen gleichzeitigen Events,
wie z.B. bei LHC-Experimenten wie ATLAS, bei denen viele Sekundärteilchen auf einmal gemessen
werden, jedes Teilchen einer Kollision zuzuweisen, damit diese einzeln analysiert werden kann.
Hierbei stellt der Schnittpunkt zwischen rekonstruierter Bahn und Strahl den Ort der
Kollision dar.
Befindet sich ein Spurdetektor zusätzlich noch in einem Magnetfeld, kann mithilfe der Lorentzkraft,
die geladene Teilchen in Bewegung auf eine Kreisbahn zwingt, die Ladung sowie der Impuls eines
Teilchens bestimmt werden.
\\
Aus den gegebenen Hits im Detektor muss, wie bereits erwähnt, die Spur eines Teilchens rekonstruiert
werden. Dabei bedient man sich bei den i.d.R. in digitaler Form vorliegenden Daten diverser
mathematischer Techniken bzw. Filter. Schwierigkeiten können vor allem bei vielen gleichzeitigen
Events und Vielfachstreuung auftreten, bei der ein Teilchen durch Wechselwirkung mit der Materie im
Detektor seine Richtung ändert. Ideale Eigenschaften eines Spurdetektors sind daher u.a. wenig
Masse und gute Orts- und Zeitauflösungen.

\section{Gasdetektoren}

Bei Gasdetektoren ist die wichtigste Art der Wechselwirkung geladener Teilchen die Ionisation. Dabei
entstehen freie Elektron-Ionen-Paare, deren Nachweis Hinweise auf die Spur des Teilchens geben
können. Der Vorteil von Gas ist dabei die hohe Mobilität der freien Ladungsträger.
In den frühesten Ausführungen von Gasdetektoren wurde die Spur optisch sichtbar gemacht und
abfotografiert, wie z.B. bei der Nebelkammer, bei der die Ionisationsspur als Kondensationsstreifen
zu verfolgen war. Ende der 1960er stellte G. Charpak die Vieldrahtkammer vor, in der die
Elektron-Ionen-Paare über eine im Gas befindliche Drahtanode bzw. Plattenkathode
gesammelt werden und diese als Stromimpuls nachgewiesen werden können. Dabei werden die Anodendrähte
in äquidistantem Abstand aufgespannt, sodass die freien Elektronen entlang der Feldlinien zum
nächstgelegenen Draht driften. Durch eine hohe Feldliniendichte nahe des Drahtes erhalten die
Elektronen durch ihre Beschleunigung genügend Energie, um weitere Elektron-Ionen-Paare zu erzeugen;
es entsteht eine lokale Ladungslawine an der Anode. Dieser verstärkte Stromimpuls ist leichter zu
messen, gleichzeitig in gewissen Spannungsbereichen aber immer noch proportional zur im Gas
deponierten Energie.
Durch mehrere Lagen von Drähten in einem Winkel zueinander verdreht, kann so durch das Ansprechen
einzelner Drähte eine mehrdimensionale Ortsinformation über ein das Gas durchquerendes Teilchen
erhalten werden.
\\
Dieses Prinzip ist die Grundlage für viele der heutigen Gasdetektoren. Auch die Driftkammer, bei der
die Driftzeit der Elektronen im Gas eine Ortsinformation liefert, funktioniert ähnlich. Durch einen
Trigger, der einen Zeitpunkt angibt, wann ein Teilchen ins Gas eintritt - die freien Elektronen also
erzeugt werden - und der Messung des Zeitpunktes, wann die Elektronen an der Drahtnode ankommen,
kann bei bekannter Driftgeschwindigkeit der Elektronen im Gas die zurückgelegte Strecke berechnet werden.
Auch hier erlauben mehrere Lagen von Anodendrähten, wenn auch weiter auseinandergelegen als bei der
Vieldrahtkammer, die Rekonstruktion der Spur.
\\
Bei MicroMegas Detektoren (Micro-Mesh Gaseous Structure) z.B., einer neueren Entwicklung, wird der
Bereich zwischen Anode und Kathode durch ein Gitter aufgeteilt, Drift- und Verstärkungsbereich. Erst
im Verstärkungsbereich über der Anode, der eine viel höhere Feldstärke aufweist als der
Driftbereich, lösen die dorthin gedrifteten Elektronen Ladungslawinen aus. Die werden über
zweilagige, senkrecht zueinander ausgerichtete Auslesestreifen aufgefangen und ergeben somit eine
zweidimensionale Ortsinformation. Mit mehreren Lagen von Micromegas kann auch hier die Spur
rekonstruiert werden.
\\
Gasdetektoren sind auch heute noch als Spurdetektoren weit verbreitet wie z.B. im Compass- oder
ATLAS- Experiment am CERN oder auch BaBar am SLAC.

\section{Halbleiterdetektoren}

Bei Halbleiterdetektoren wird grundsätzlich die Raumladungszone eines pn-Übergangs einer Diode im
Sperrbetrieb zum Nachweis geladener Teilchen verwendet. Tritt ein solches Teilchen in diese
Sperrschicht ein und gibt Energie an Valenzlektronen ab, so können diese in das Leitungsband
wechseln. Zurück bleibt ein Loch bzw. Defektelektron. Beide können als quasifreie Ladungsträger zur
Anode bzw. Kathode wandern und dort wiederum als Stromimpuls gemessen werden. Auch hierbei ist die
Anzahl der erzeugten Elektron-Loch-Paare proportional zur Energie, die das Teilchen im Material
abgegeben hat.
Der Aufbau erfolgt als Streifen- oder Pixeldetektoren. Im ersten Fall werden dabei das i.d.R. 
p-dotiere Material streifenförmig auf das n-dotierte Grundmasse aufgebracht, ein doppelseitiger
Aufbau ermöglicht eine zweidimensionale Zuordnung der Treffer. Auch bei in Rechtecken segmentierten
p-Elektroden ist dies möglich. Mehrere Detektoren hintereinander lassen eine Spurrekonstruktion
zu.\\
Halbleiterdetektoren werden wegen guter Ortsauflösung oft als erste Schicht am Kollisionspunkt
eingesetzt, wie beispielsweise beim ATLAS-Experiment am CERN.

\section{Szintillationszähler}

Szintillatoren bestehen aus einem Material, dessen Moleküle nach Anregung durch geladene Teilchen
oder Photonen die Anregungsenergie in Form von Licht wieder abgeben. Das Material für
Szintillationsdetektoren ist meistens organischer Natur, wie z.B. Plastikszintillatoren.
Szintillationszähler sind dünne Szintillationsfasern, die aufeinander geschichtet bzw. gebündelt
werden. Wird Licht durch ein hindurchgehendes Teilchen erzeugt, wird dies über Totalreflexion durch 
die Faser weitergeleitet und am Ende durch Vielkanalphotomultiplier in ein elektrisches Signal
umgewandelt.
\\
Die Spur kann rekonstriert werden, indem mehrere Faserlagen in einem Winkel zueinander in Schichten
angeordnet werden. Szintillationszähler werden zum Beispiel am DZero-Experiment am Fermilab
eingesetzt sowie am A1 Experiment am MaMi.

\section{Zusammenfassung}

Die Frage, welcher Typ von Spurdetektor für welches Experiment am geeignetsten ist, lässt sich
pauschal kaum beantworten. Zwar erscheint eine gute Ortsauflösung in jedem Falle wünschenswert,
jedoch spielen eine Menge Faktoren in der Wahl der Detektorart eine Rolle: Ist die maximal
erreichbare Ortsauflösung für den Zweck eines Experiments überhaupt notwendig? Wieviel Platz ist
verfügbar, wie sollte der Detektor geometrisch angeordet sein? Stehen die Mittel und die
erforderliche Zeit zur Verfügung, um einen bestimmten Detektor bauen zu können?
\\
Gasdetektoren beispielsweise sind im Aufbau verhältnismäßig günstig und können großflächig aufgebaut
werden, weiterhin liegt das Signal direkt in elektronischer Form vor. Nicht nur die je nach Variante
mäßige Ortsauflösung und hohe Ansprechzeit, auch die Infrastruktur wie Gaszuleitungen etc. müssen
jedoch berücksichtigt werden. Halbleiterdetektoren weisen die wohl beste Ortsauflösung auf und liefern
ebenfalls direkt ein elektronisches Signal, weisen allerdings auch eine hohe Verlustleistung auf,
eine begrenzte Lebensauer bei hoher Strahlung und hohe Anschaffungskosten. Szintillationszähler
haben wenig Masse und produzieren daher wenig Vielfachstreuung, haben eine kleine Ansprechzeit und
sind sehr flexibel, jedoch ebenfalls kostspielig. Die schlechte Lichtausbeute kann auch dazu führen,
dass der Nachweis der Photonen mit Schwierigkeiten verbunden ist.
\\
Nicht alle Ausführungen von Spurdetektoren konnten hier behandelt werden. Weiterhin ist dies
natürlich ein Gebiet, auf dem ständig geforscht wird, sodass zukünftig vielleicht noch andere
Methoden gefunden werden, um die Spuren von Teilchen sinnvoll zu rekonstruieren.


% \bibliography{biblio}{}
% \bibliographystyle{unsrt}
% \printbibliography

\section*{Referenzen}
\begin{description}
\item[\ensuremath{[1]}]~~W. R. Leo. \textit{Techniques for Nuclear and Particle Physics
Experiments}. Springer-Verlag, 1994.
\item[\ensuremath{[2]}]~~Claude Leroy and Pier-Giorgio Rancoita. \textit{Principles of Radiation Interaction in Matter
and Detection}. World Scientific, 2009.
\item[\ensuremath{[3]}]~~Dan Green. \textit{The Physics of Particle Detectors}. Cambridge University Press,
2005.
\item[\ensuremath{[4]}]~~ATLAS Experiment. Siehe cern.ch/ATLAS.
\item[\ensuremath{[5]}]~~Compass Experiment. Siehe cern.ch/Compass.
\item[\ensuremath{[6]}]~~BaBar Experiment. Siehe http://www-public.slac.stanford.edu/babar.
\item[\ensuremath{[7]}]~~DZero Experiment. Siehe http://www-d0.fnal.gov/
\item[\ensuremath{[8]}]~~A1 Experiment. Siehe http://wwwa1.kph.uni-mainz.de/A1/
\end{description}
\end{document}